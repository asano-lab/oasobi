\documentclass[a4paper,11pt]{jsarticle}


% 数式
\usepackage{amsmath,amsfonts, amssymb}
\usepackage{bm}
% 画像
\usepackage[dvipdfmx]{graphicx}


\begin{document}

\title{巡回符号の課題}
\author{}
\date{\today}
\maketitle


\section{$n=5$の巡回符号}
生成多項式$g(x)$は$1+x^n$の因数なので、
\[
  1+x^5=(1+x)(1+x+x^2+x^3+x^4)
\]
より、$g(x)$の候補は$1+x$と$1+x+x^2+x^3+x^4$
である。
$g(x)=1+x$のときの全ての符号を表\ref{table:n5g1}
に示す。
\begin{table}[hbtp]
  \caption{$g(x)=1+x$の符号}
  \label{table:n5g1}
  \centering
  \begin{tabular}{ccccc}
    00000 &&&& \\
    00011 & 00110 & 01100 & 11000 & 10001 \\
    00101 & 01010 & 10100 & 01001 & 10010 \\
    01111 & 11110 & 11101 & 11011 & 10111
  \end{tabular}
\end{table}
表\ref{table:n5g1}より、$g(x)=1+x$のときの最小ハミング距離は2である。
$g(x)=1+x+x^2+x^3+x^4$のとき、符号は00000と11111のみなので、
最小ハミング距離は5である。

\section{$n=6$の巡回符号}
生成多項式$g(x)$は$1+x^n$の因数なので、
\[
  1+x^6=(1+x^3)^2=(1+x)^2(1+x+x^2)^2
\]
より、$g(x)$の候補を$k$の昇順に並べると
表\ref{table:n6g}になる。
\begin{table}[hbtp]
  \caption{$n=6$の巡回符号の生成多項式}
  \label{table:n6g}
  \centering
  \begin{tabular}{c|l}
    $k$ & $g(x)$ \\ \hline
    1 & $(1+x)(1+x+x^2)^2$ \\
    2 & $(1+x+x^2)^2$ \\
    2 & $(1+x)^2(1+x+x^2)$ \\
    3 & $(1+x)(1+x+x^2)$ \\
    4 & $1+x+x^2$ \\
    4 & $(1+x)^2$ \\
    5 & $1+x$
  \end{tabular}
\end{table}

\section{$g(x)=(1+x)(1+x^2+x^3)$の巡回符号の生成行列}
$n=7$の場合、$g(x)$の次数が4なので、情報源の長さは3である。
$g(x)$を展開すると$1+x+x^2+x^4$となり、
生成行列は1行目が$g(x)$, 2行目が$xg(x)$, 3行目が$x^2g(x)$
になるので、
\[
  {\bm G}=
  \begin{bmatrix}
    0 & 0 & 1 & 1 & 1 & 0 & 1 \\
    0 & 1 & 1 & 1 & 0 & 1 & 0 \\
    1 & 1 & 1 & 0 & 1 & 0 & 0
  \end{bmatrix}
\]
となる。これを組織符号に変換すると
\[
  \begin{bmatrix}
    0 & 1 & 1 \\
    1 & 1 & 0 \\
    1 & 0 & 0
  \end{bmatrix}
  \begin{bmatrix}
    0 & 0 & 1 & 1 & 1 & 0 & 1 \\
    0 & 1 & 1 & 1 & 0 & 1 & 0 \\
    1 & 1 & 1 & 0 & 1 & 0 & 0
  \end{bmatrix}=
  \begin{bmatrix}
    1 & 0 & 0 & 1 & 1 & 1 & 0 \\
    0 & 1 & 0 & 0 & 1 & 1 & 1 \\
    0 & 0 & 1 & 1 & 1 & 0 & 1
  \end{bmatrix}
\]


\end{document}