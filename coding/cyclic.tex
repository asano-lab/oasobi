\documentclass[a4paper,11pt]{jsarticle}


% 数式
\usepackage{amsmath,amsfonts, amssymb}
\usepackage{bm}
% 画像
\usepackage[dvipdfmx]{graphicx}


\begin{document}

\title{巡回符号の課題}
\author{}
\date{\today}
\maketitle


\section{$n=5$の巡回符号}
生成多項式$g(x)$は$1+x^n$の因数なので、
\[
  1+x^5=(1+x)(1+x+x^2+x^3+x^4)
\]
より、$g(x)$の候補は$1+x$と$1+x+x^2+x^3+x^4$
である。
$g(x)=1+x$のときの全ての符号を表\ref{table:n5g1}
に示す。
\begin{table}[hbtp]
  \caption{$g(x)=1+x$の符号}
  \label{table:n5g1}
  \centering
  \begin{tabular}{ccccc}
    00000 &&&& \\
    00011 & 00110 & 01100 & 11000 & 10001 \\
    00101 & 01010 & 10100 & 01001 & 10010 \\
    01111 & 11110 & 11101 & 11011 & 10111
  \end{tabular}
\end{table}
表\ref{table:n5g1}より、$g(x)=1+x$のときの最小ハミング距離は2である。
$g(x)=1+x+x^2+x^3+x^4$のとき、符号は00000と11111のみなので、
最小ハミング距離は5である。

\section{$n=6$の巡回符号}
生成多項式$g(x)$は$1+x^n$の因数なので、
\[
  1+x^6=(1+x^3)^2=(1+x)^2(1+x+x^2)^2
\]
より、$g(x)$の候補を$k$の昇順に並べると
表\ref{table:n6g}になる。
\begin{table}[hbtp]
  \caption{$n=6$の巡回符号の生成多項式}
  \label{table:n6g}
  \centering
  \begin{tabular}{c|l}
    $k$ & $g(x)$ \\ \hline
    1 & $(1+x)(1+x+x^2)^2$ \\
    2 & $(1+x+x^2)^2$ \\
    2 & $(1+x)^2(1+x+x^2)$ \\
    3 & $(1+x)(1+x+x^2)$ \\
    4 & $1+x+x^2$ \\
    4 & $(1+x)^2$ \\
    5 & $1+x$
  \end{tabular}
\end{table}

\section{$g(x)=(1+x)(1+x^2+x^3)$の巡回符号の生成行列}
$n=7$の場合、$g(x)$の次数が4なので、情報源の長さは3である。
$g(x)$を展開すると$1+x+x^2+x^4$となり、
生成行列は1行目が$x^2g(x)$, 2行目が$xg(x)$, 3行目が$g(x)$
になるので、
\[
  {\bm G}=
  \begin{bmatrix}
    0 & 0 & 1 & 1 & 1 & 0 & 1 \\
    0 & 1 & 1 & 1 & 0 & 1 & 0 \\
    1 & 1 & 1 & 0 & 1 & 0 & 0
  \end{bmatrix}
\]
となる。これを組織符号に変換すると
\[
  \begin{bmatrix}
    0 & 1 & 1 \\
    1 & 1 & 0 \\
    1 & 0 & 0
  \end{bmatrix}
  \begin{bmatrix}
    0 & 0 & 1 & 1 & 1 & 0 & 1 \\
    0 & 1 & 1 & 1 & 0 & 1 & 0 \\
    1 & 1 & 1 & 0 & 1 & 0 & 0
  \end{bmatrix}=
  \begin{bmatrix}
    1 & 0 & 0 & 1 & 1 & 1 & 0 \\
    0 & 1 & 0 & 0 & 1 & 1 & 1 \\
    0 & 0 & 1 & 1 & 1 & 0 & 1
  \end{bmatrix}
\]

\section{$g(x)=1+x^2+x^3$の(7,4)巡回符号}
\subsection{符号化回路とその動作}
$g(x)=1+x^2+x^3$の(7,4)巡回符号の符号化回路
を図\ref{fig:encoding-circuit}に示す。
\begin{figure}[htbp]
  \begin{center}
  \includegraphics[scale=1.0]{figures/cyclic7-4-encoding.pdf}
  \end{center}
  \caption{符号化回路
  \label{fig:encoding-circuit}
  }
\end{figure}
この回路に入力情報$u(x)=1+x^2$を
入れたときの動作を表\ref{table:circuit-behavior}に示す。
情報の長さは4なので、入力の順番は0101である。
\begin{table}[hbtp]
  \caption{$u(x)=1+x^2$を入れたときの動作}
  \label{table:circuit-behavior}
  \centering
  \begin{tabular}{cc|ccc|c}
    & 入力 & $b_0$ & $b_1$ & $b_2$ & 出力 \\ \hline
    & - & 0 & 0 & 0 & - \\
    & 0 & 0 & 0 & 0 & 0 \\
    SW1: 閉 & 1 & 1 & 0 & 1 & 1 \\
    SW2: 上 & 0 & 1 & 1 & 1 & 0 \\
    & 1 & 0 & 1 & 1 & 1 \\ \hline
    SW1: 開 & - & 0 & 0 & 1 & 1 \\
    SW2: 下 & - & 0 & 0 & 0 & 1 \\
    & - & 0 & 0 & 0 & 0
  \end{tabular}
\end{table}

\subsection{シンドローム計算回路}
シンドローム計算回路を図\ref{fig:calc-syndrome-circuit}に示す。
\begin{figure}[htbp]
  \begin{center}
  \includegraphics[scale=1.0]{figures/calc-syndrome-cyclic-7-4.pdf}
  \end{center}
  \caption{シンドローム計算回路
  \label{fig:calc-syndrome-circuit}
  }
\end{figure}
受信語が$r(x)=x^2+x^4+x^5$の場合の
回路の動作を表\ref{table:syndrome-behavior}に示す。
\begin{table}[hbtp]
  \caption{$r(x)=x^2+x^4+x^5$を入れたときの動作}
  \label{table:syndrome-behavior}
  \centering
  \begin{tabular}{c|ccc}
    入力 & $s_0$ & $s_1$ & $s_2$ \\ \hline
    - & 0 & 0 & 0 \\
    0 & 0 & 0 & 0 \\
    1 & 1 & 0 & 0 \\
    1 & 1 & 1 & 0 \\
    0 & 0 & 1 & 1 \\
    1 & 0 & 0 & 0 \\
    0 & 0 & 0 & 0 \\
    0 & 0 & 0 & 0
  \end{tabular}
\end{table}
シンドロームが最終的に0になったので誤りなし。
また、$r(x)=x^2g(x)$であることからも、
誤りがないことが分かる。

\section{$g(x)=1+x+x^4$の(15, 11)ハミング符号}
符号化回路を図\ref{fig:hamming15-11-encoding}に示す。
\begin{figure}[htbp]
  \begin{center}
  \includegraphics[scale=1.0]{figures/hamming15-11-encoding.pdf}
  \end{center}
  \caption{$g(x)=1+x+x^4$の(15, 11)ハミング符号の符号化回路
  \label{fig:hamming15-11-encoding}
  }
\end{figure}

復号回路では$x^{n-1}$のシンドロームだけ探せばよいので、
\[
  x^{14}=(1+x+x^4)(1+x+x^2+x^4+x^6+x^7+x^{10})+(1+x^3)
\]
より、$1+x^3$で1になる論理回路を使う。
復号回路を図\ref{fig:hamming15-11-decoding}に示す。
\begin{figure}[htbp]
  \begin{center}
  \includegraphics[scale=0.6]{figures/hamming15-11-decoding.pdf}
  \end{center}
  \caption{$g(x)=1+x+x^4$の(15, 11)ハミング符号の復号回路
  \label{fig:hamming15-11-decoding}
  }
\end{figure}

\section{生成多項式が$g(x)$、長さnの符号}
\subsection{$(1+x)$が$g(x)$の因数であれば、奇数重みの符号語が存在しないことを示せ}
$g(x)$が$(1+x)$を因数に持つとき、$g(x)=(1+x)a(x)$と表せる。
$g(x)$に$x=1$を代入すると
\[
  g(1)=(1+1)a(1)=0
\]
となる。符号語$w(x)$は$g(x)$で割り切れるので$w(x)=b(x)g(x)$と表せ、
\[
  w(1)=b(1)g(1)=0
\]
となる。1を代入して0になる多項式は項数が偶数なので、
符号語は必ず偶数重みになる。
つまり、$(1+x)$が$g(x)$の因数であれば、
奇数重みの符号語が存在しない。

\subsection{$n$が奇数で、$(1+x)$が$g(x)$の因数でない場合、符号語の一つは (111...1) であることを示せ。}
$(1+x)$は$(1+x^n)$の因数であり、
\[
  1+x^n=(1+x)(1+x+x^2+ \cdots +x^{n-1})
\]
と因数分解できる。$g(x)$は$(1+x^n)$の因数であるため、
$(1+x)$が$g(x)$の因数でないとき、$g(x)$は
$(1+x+x^2+ \cdots + x^{n-1})$の因数である。
$g(x)$で割り切れる多項式は符号語なので、
$(1+x+x^2+ \cdots + x^{n-1})$ (ベクトル表現: 111...1)も符号語である。

\subsection*{補足}
\begin{eqnarray*}
  \label{eq1}
  1+x^n&=&(1+x)(1+x+x^2+ \cdots + x^{n-1})\\
  &=&(1+x)a(x)b(x)c(x) \cdots
\end{eqnarray*}
であり、$(1+x)$が$g(x)$の因数でない。つまり$g(x)$は$(1+x)$で
割り切れず、
\begin{equation}
  \label{eq2}
  g(x) \neq (1+x)q(x)
\end{equation}
なので、

\begin{table*}[hbtp]
  \label{table:hissan}
  \centering
  \begin{tabular}{ccccccccc}
    & $s(x)$ & = & 1 & $+x$ & $+x^2$ & $+\cdots$ & $+x^{n-1}$ & \\
    +) & $xs(x)$ & = & & $x$ & $+x^2$ & $+\cdots$ & $+x^{n-1}$ & $+x^{n}$ \\ \hline
    & $(1+x)s(x)$ & = & 1 & & & & & $+x^n$
  \end{tabular}
\end{table*}

\subsection{$(1+x^i)$が$g(x)$で割り切れる一番小さい値$i$が$n$である場合、 最小ハミング距離が3以上であることを示せ。}
$(1+x^i)$が$g(x)$で割り切れるような$i$の最小値を$g(x)$の周期という。
$g(x)$の周期を$n$とし、ハミング重みが2の符号が存在すると仮定する。


\end{document}